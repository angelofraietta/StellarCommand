\chapter{Introduction}
    Musical composition and performance inspired or based on astronomy has been used in many cultures for millennia, with many civilizations creating songs and dances based on the astronomical calendar to reinforce the tracking of seasonal activities; such as planting and harvesting of crops, times of trade, and religious or cultural practices \cite{ruggles2015handbook, deMello2015, Lima2015}.  More recently, composers  have used scientific data obtained from individual stars to generate sounds and have created compositions directly correlated to that data \cite{fraietta2014musical, BriightSyzygy}. 

The advancement of computing power has made the availability of planetarium software for both desktop and mobile platforms very accessible to many people. These software packages are not only used for scientific and research activities; such as astronomy, education and general stargazing; they have also been used by artists in presenting multimedia artworks and installations \cite{zotti2017skyscape,tuveri2013controlling}. 

Many composers have used the cosmos as inspiration or stimulus to their works, with many using scientific observations or data as input \cite{fraknoi2008music, fraietta2014musical}. The \textit{Quadrivium} linked astronomy, mathematics, geometry and music as a standard part of classical education up until the renaissance \cite{lundy2010quadrivium}. Composers have been mapping mathematics and geometry to music since antiquity \cite{james1995music, assayag2002mathematics}. Kepler stated ``The heavenly motions are nothing but a continuous song for several voices, to be perceived by the intellect, not by the ear; a music which, through discordant tensions, through syncopations and cadenzas as it were, progresses toward certain pre designed six-voiced cadences, and thereby sets landmarks in the immeasurable flow of time." \cite[cited in  ~286]{RojersRuffKepler}. This notion inspired Rogers and Ruff to compose \textit{The Harmony of the World} (1979), describing their work as ``A Realization for the Ear" \cite [p. 286]{RojersRuffKepler} of Johannes Kepler's Astronomical Data from Harmonices Mundi 1619.
More recently, composers have used measurements from online databases as inputs to automata or as stimulus to performers. 


I originally developed a system based on astronomical catalogues for musical composition and performance interface using naked eye and binocular astronomy \cite{fraietta2014musical}. A specific star was determined by calculating its azimuth and height above the horizon--known as its \textit {altitude}--using  accelerometer and magnetometer sensors, and calculating the exact location of the star on the celestial sphere using the sensor data, time and geographical location of the observer. This calculation returns the star's \textit{right ascension (RA)}, which is based on its azimuth at Greenwich Meantime at the vernal equinox,  and its \textit{declination (Dec.)}, which is the stars north-south position at the same time \cite{duffett2011practical, fraietta2014musical}. The resultant RA and Dec. are added as input to the VizieR database of online catalogues, returning data about stars--such as brightness and colour--within the defined radius. Various works were created using this interface. In one performance, which was  conducted in conjunction with the Newcastle Astronomical Society on one of their field viewing nights, members of the  public were enticed into viewing the night sky through high powered binoculars, while the sound generated, which was  based on data from the stars they were viewing, was played through loudspeakers on the field \cite{fraietta_segue}. 
Another set of performances was conducted with an improvising ensemble that featured various astronomical photos displayed as a slide show where the astronomical data was mapped as MIDI and functioned as inspirational impetus for the performers \cite{BriightSyzygy}. 

Although the binocular display has an awesome display--the actual night sky--``few people ventured outside to the astronomical equipment"\cite[p. ~50]{fraietta2014musical} because they were required to leave the room to look through binoculars while the ensemble played in a room. Furthermore, the work is severely bound by weather conditions and a clear view of the sky. In one of the performances, ``the sky was completely covered with cloud and it rained, so there was nothing to see through the binoculars. The audience, however, enjoyed the ensemble performance with the NASA image slide show with samples fed from stored star tables."\cite[p. ~50]{fraietta2014musical}.  This weather constraint inspired me to use planetarium software as the input and display mechanism as an alternative to binoculars.  

Stellar Command enables composers to access astronomical data as input to their software using a common API.  It also enables you to provide the audience an impressive planetarium software display that runs on a laptop computer without requiring the audience to leave the room. Furthermore, it facilitates creation of interactive celestial based installations. Stellar Command is available as open source software through GitHub \cite{fraiettaSTELLARCOMMAND}.
